\documentclass[]{book}
\usepackage{lmodern}
\usepackage{amssymb,amsmath}
\usepackage{ifxetex,ifluatex}
\usepackage{fixltx2e} % provides \textsubscript
\ifnum 0\ifxetex 1\fi\ifluatex 1\fi=0 % if pdftex
  \usepackage[T1]{fontenc}
  \usepackage[utf8]{inputenc}
\else % if luatex or xelatex
  \ifxetex
    \usepackage{mathspec}
  \else
    \usepackage{fontspec}
  \fi
  \defaultfontfeatures{Ligatures=TeX,Scale=MatchLowercase}
\fi
% use upquote if available, for straight quotes in verbatim environments
\IfFileExists{upquote.sty}{\usepackage{upquote}}{}
% use microtype if available
\IfFileExists{microtype.sty}{%
\usepackage{microtype}
\UseMicrotypeSet[protrusion]{basicmath} % disable protrusion for tt fonts
}{}
\usepackage[margin=1in]{geometry}
\usepackage{hyperref}
\hypersetup{unicode=true,
            pdftitle={Statistical and Population Genetics},
            pdfauthor={Hanbin Lee},
            pdfborder={0 0 0},
            breaklinks=true}
\urlstyle{same}  % don't use monospace font for urls
\usepackage{natbib}
\bibliographystyle{apalike}
\usepackage{longtable,booktabs}
\usepackage{graphicx,grffile}
\makeatletter
\def\maxwidth{\ifdim\Gin@nat@width>\linewidth\linewidth\else\Gin@nat@width\fi}
\def\maxheight{\ifdim\Gin@nat@height>\textheight\textheight\else\Gin@nat@height\fi}
\makeatother
% Scale images if necessary, so that they will not overflow the page
% margins by default, and it is still possible to overwrite the defaults
% using explicit options in \includegraphics[width, height, ...]{}
\setkeys{Gin}{width=\maxwidth,height=\maxheight,keepaspectratio}
\IfFileExists{parskip.sty}{%
\usepackage{parskip}
}{% else
\setlength{\parindent}{0pt}
\setlength{\parskip}{6pt plus 2pt minus 1pt}
}
\setlength{\emergencystretch}{3em}  % prevent overfull lines
\providecommand{\tightlist}{%
  \setlength{\itemsep}{0pt}\setlength{\parskip}{0pt}}
\setcounter{secnumdepth}{5}
% Redefines (sub)paragraphs to behave more like sections
\ifx\paragraph\undefined\else
\let\oldparagraph\paragraph
\renewcommand{\paragraph}[1]{\oldparagraph{#1}\mbox{}}
\fi
\ifx\subparagraph\undefined\else
\let\oldsubparagraph\subparagraph
\renewcommand{\subparagraph}[1]{\oldsubparagraph{#1}\mbox{}}
\fi

%%% Use protect on footnotes to avoid problems with footnotes in titles
\let\rmarkdownfootnote\footnote%
\def\footnote{\protect\rmarkdownfootnote}

%%% Change title format to be more compact
\usepackage{titling}

% Create subtitle command for use in maketitle
\newcommand{\subtitle}[1]{
  \posttitle{
    \begin{center}\large#1\end{center}
    }
}

\setlength{\droptitle}{-2em}

  \title{Statistical and Population Genetics}
    \pretitle{\vspace{\droptitle}\centering\huge}
  \posttitle{\par}
    \author{Hanbin Lee}
    \preauthor{\centering\large\emph}
  \postauthor{\par}
      \predate{\centering\large\emph}
  \postdate{\par}
    \date{2019-07-18}

\usepackage{booktabs}

\usepackage{amsthm}
\newtheorem{theorem}{Theorem}[chapter]
\newtheorem{lemma}{Lemma}[chapter]
\theoremstyle{definition}
\newtheorem{definition}{Definition}[chapter]
\newtheorem{corollary}{Corollary}[chapter]
\newtheorem{proposition}{Proposition}[chapter]
\theoremstyle{definition}
\newtheorem{example}{Example}[chapter]
\theoremstyle{definition}
\newtheorem{exercise}{Exercise}[chapter]
\theoremstyle{remark}
\newtheorem*{remark}{Remark}
\newtheorem*{solution}{Solution}
\let\BeginKnitrBlock\begin \let\EndKnitrBlock\end
\begin{document}
\maketitle

{
\setcounter{tocdepth}{1}
\tableofcontents
}
\newcommand{\E}{\mathbb{E}}
\newcommand{\V}{\mathrm{Var}}

\chapter{Prerequisites}\label{prerequisites}

Elementary probability theory and linear algebra will be sufficient.
However, topics regarding theoretical genetics requires knowledge about
stochastic processes and measure theory.

Reference on advanced mathematics can be found in
\citet{walterrudin1986}, \citet{rickdurrett2019} and
\citet{rickdurrett2008}.

\chapter{Introduction}\label{intro}

As a mathematics undergraduate, theorem-proof structure was much easier
to understand. However, most statistical \& population genetics articles
were less familiar to me in their organization.

This page aims to list theorems and methods regarding statistical \&
population genetics in a theorem-proof structure so that I can readily
access when I need them.

\chapter{Linkage Disequilibrium Score Regression
(LDSC)}\label{linkage-disequilibrium-score-regression-ldsc}

Linkage Disequilibrium Score Regression (LDSC) is a popular method in
statistical genetics with a wide range of application. It is used to
measure confounding due to population structure, computing genetic
correlation and heritability. Most of all, the method can be performed
soley on summary statistics rather than the full data which reduces
memeory and computation requirements tremendously.

\section{Univariate LDSC}\label{univariate-ldsc}

The original from of LDSC (here we call it Univariate LDSC) can be used
to estimate SNP heritability and confounding effects of population
structure \citep{BulikSullivan2015}. I refer the supplementary notes of
\citet{BulikSullivan2015} for the notations.

\BeginKnitrBlock{definition}[LD score]
\protect\hypertarget{def:unnamed-chunk-1}{}{\label{def:unnamed-chunk-1}
\iffalse (LD score) \fi{} }Define the \textbf{LD Score} of variant \(j\)
as \[
  l_j = \sum_{k=1}^{M} r_{jk}^2
\]
\EndKnitrBlock{definition}

Then for a homogenous sample without population structure, the following
holds.

\BeginKnitrBlock{theorem}[LDSC without structure]
\protect\hypertarget{thm:ldsc1}{}{\label{thm:ldsc1} \iffalse (LDSC without
structure) \fi{} }In a sample without population structure \[
  \E[\chi_j^2] = \frac{Nh_g^2}{M} l_j +1 
\] holds.
\EndKnitrBlock{theorem}

If a population structure (measured by the fixation index \(F_{ST}\))
exists, a non-zero constant is added to \label{thm:ldsc1}.

\BeginKnitrBlock{theorem}[General LDSC]
\protect\hypertarget{thm:ldsc2}{}{\label{thm:ldsc2} \iffalse (General LDSC)
\fi{} }In a sample with population structure \[
  \E[\chi_j^2] = \frac{Nh_g^2}{M} l_j + 1 + aNF_{ST}
\] holds.
\EndKnitrBlock{theorem}

\BeginKnitrBlock{proof}
\iffalse{} {Proof. } \fi{} \[\begin{aligned}
  \V[\hat{\beta_j}] &= \E[\V[\hat{\beta_j}|X]]+\V[\E[\hat{\beta_j}|X]]  \\
  &= \E[\V[\hat{\beta_j}|X]] + 0 \\
\end{aligned}\]

The second term vanishes since \(\E[\hat{\beta_j}|X]= 0\).

Inspecting the first term, \[\begin{aligned}
  \V[\hat{\beta_j}|X] &= \frac{1}{N^2} \V[X_j^T \phi|X] \\
  &= \frac{1}{N^2} X_j^T \V[\phi|X] X_j \\
  &= \frac{1}{N^2} \left( \frac{h_g^2}{M}X_j^T XX^T X_j + N(1-h_g^2) \right)
\end{aligned}\]
\EndKnitrBlock{proof}

\section{Bivariate LDSC}\label{bivariate-ldsc}

\section{Partitioned LDSC}\label{partitioned-ldsc}

\chapter{Mendelian Randomization}\label{mendelian-randomization}

Mendelian Randomization (MR) is a special case of the Instrumental
Variable (IV) which is widely used in econometrics. MR employs genetic
variants (e.g.~SNP) as IVs since genotypes are automatically randomized
in the gametic phase of cell division.

\section{Instrumental Variable (IV)}\label{instrumental-variable-iv}

A \(l \times 1\) random vector \(\mathbf{z_i}\) is an instrumental
variable if

\begin{enumerate}
\def\labelenumi{\arabic{enumi}.}
\tightlist
\item
  \(\E(\mathbf{z_i} e_i) = 0\)
\item
  \(\E(\mathbf{z_i z_i} ^T) = 0\)
\item
  \(\mathrm{rank} \, \E(\mathbf{z_i x_i}^T) = k\)
\end{enumerate}

holds \citep{brucehansen}.

From this we have \[\begin{aligned}
  0 &= \E[\mathbf{z_i} e_i] \\
    &= \E[\mathbf{z_i} (y_i - \mathbf{x_i}^T\mathbf{\beta})] \\
    &= \E[\mathbf{z_i}y_i] - \E[\mathbf{z_ix_i}^T] \mathbf{\beta}
\end{aligned}\]

Rearranging the above gives \[
\mathbf{\beta} = \E[\mathbf{z_ix_i}^T] ^{-1} \E[\mathbf{z_i} y_i]
\]

suggesting the \textbf{IV estimator} \[
\mathbf{\hat{\beta}_{iv}} = (Z^TX)^{-1}(Z^T\mathbf{y})
\] that is consistent at a large \(n \rightarrow \infty\) limit.

\section{Simple MR}\label{simple-mr}

\section{Two-sample MR}\label{two-sample-mr}

\section{MR-Egger}\label{mr-egger}

\section{Weighted Median MR}\label{weighted-median-mr}

\section{Modal Based MR}\label{modal-based-mr}

\bibliography{book.bib,packages.bib}


\end{document}
